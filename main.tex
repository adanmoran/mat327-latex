%! TEX program = lualatex

% This project should be between 5-6 pages.
% No cover page required.
% Font should be 12pt, with 3/4" margins on all sides.
% It should be properly written, divided into sections, 
% and must contain a proper introduction and bibliography.

\documentclass[12pt]{article}
%/========== Preamble ==========/%
% Set Margins to 3/4"
\usepackage[margin=0.75in]{geometry}

%---------- Math Packages ----------%
\usepackage{amsmath}
\usepackage{amssymb}
\usepackage{mathtools}
\usepackage{amsthm}

% Define theorem styles
\newtheorem{thm}{Theorem}

\theoremstyle{definition}
\newtheorem{defn}{Definition}
\newtheorem{example}{Example}

%---------- Bibliography Style ----------%
\usepackage[style=ieee,backend=biber]{biblatex}
\addbibresource{bib.bib}

%/========== Main Document ==========/%
\begin{document}
    
\section{Introduction}
One common approach to controlling robotic manipulators 
(known colloquially as ``robot arms") is to pick a pose of the end effector and
reverse-engineer the joint angles required to reach this pose. This is known as
the \textbf{inverse kinematics} problem. For an arm with \(n\) revolute joints, 
one can model the desired end effector position and orientation by
\(x \in \mathbb{R}^3 \times SO(3)\) and the angles of the joints by 
\(\theta \in \mathbb{T}^n\) \cite{robots-fiber-bundles}. 
The inverse kinematics problem requires solving the
equation \(f(\theta) = x\) for \(\theta\), where \(f\) is an 
appropriately-defined function \cite{program-kin-red-manips}.

Unfortunately, there always exist joint angles where the relationship 
\(x = f(\theta)\) breaks down. These points are called 
\textbf{singular configurations}. If the dimension of the workspace is 
\(k > 0\), the singular configurations are characterized by the set
\[
    S = \left\{ \theta_0 \in \mathbb{T}^n \mid 
    \text{rank}\left\{
        \frac{\partial f}{\partial\theta}(\theta_0) 
    \right\} < k \right\}
\]

\begin{example}
%TODO: the 2D planar manipulator image.

The 2D planar manipulator with two revolute joints of lengths \(r_1 > 0\)
and \(r_2 > 0\) (Figure ????) whose end-effector is fixed in orientation with the
second joint has position \((x,y) \in \mathbb{R}^2\) and joint angles 
\(\theta = (\theta_1,\theta_2) \in \mathbb{T}^2\). 
The relation connecting these two is given by (\ref{eqn:2d-planar-manip}).
\begin{equation}\label{eqn:2d-planar-manip}
    \begin{bmatrix}x \\ y \end{bmatrix}
    = \begin{bmatrix} 
        r_1\cos(\theta_1) + r_2\cos(\theta_1 + \theta_2) \\
        r_1\sin(\theta_1) + r_2\sin(\theta_1 + \theta_2)
    \end{bmatrix} =: f(\theta)
\end{equation}

The Jacobian has determinant
\[
    \left| \frac{\partial f}{\partial \theta} \right| = r_1r_2\sin(\theta_2)
\]
which means the singular configurations are 
\(S = \left\{ (\theta_1,\theta_2) \in \mathbb{T}^2 \mid \theta_2 = \pm\pi\right\}\); 
that is, the singular configurations of this robot arm are exactly when the
second joint is colinear with the first joint.
\end{example}

Singularities cause various issues when controlling robot arms, as the
controllers will demand infinite torque be applied to track specific positions
or velocities when near these configurations. For this reason, it is preferable
to avoid singular configurations when possible. In particular, when generating a
trajectory for the end-effector, a good control mechanism should avoid choosing
joint positions which land ``close to" the singular configurations.

Previous researchers have tried to solve this problem using extra limbs
\cite{program-kin-red-manips} and velocity constraints
\cite{articulated-robot-redundancy}, among other approaches involving kinematic
models of the robots. The authors of
``Topology and the Robot Arm" offer a different approach: they ask whether
avoiding kinematic singularities is possible at all, and prove when it can be
done using the topology of the robot's configuration space
\cite{topology-robot-arm}.

This report will cover the relevant background required to understand 
``Topology and the Robot Arm",
and will summarize the results of the paper.

\section{Relevant Background}
\begin{defn}[Fiber bundles \cite{topology-fiber-bundles}]
    Let \(E\), \(X\), and \(F \subset E\) be topological spaces, 
    with \(X\) connected.
    Let \(f : E \rightarrow X\) be a continous surjective function. 
    We say \(f\) is a \textbf{locally trivial fibration} or a 
    \textbf{fiber bundle with fiber \(F\)} and write 
    \(F \rightarrow E \xrightarrow[]{f} X\) when
    \begin{enumerate}
        \item \(f^{-1}(\{x_0\}) = F \, \forall x_0 \in X\)
        \item \((\forall x \in X)\) there is an open neighbourhood of \(x\)
            \(U_x \subset X\) and a homeomorphism
            \(\psi_x : f^{-1}(U_x) \rightarrow U_x \times F\) so that
            \(f\vert_{f^{-1}(U_x)} = \pi_1 \circ \psi_x\) (where \(\pi_1\)
            is the projection map of \(U_x \times F\) onto \(U_x\))
    \end{enumerate}
    Note that it is common to say that \(E\) itself is the fiber bundle over
    \(X\) if \(E = X \times F\), since the natural projection 
    \(\pi : X \times F \rightarrow X\) is a fiber bundle.
\end{defn}

\begin{defn}[Vector bundles \cite{topology-fiber-bundles}]
    Let \(V \rightarrow E \xrightarrow[]{f} E\) be a fiber bundle and \(V\) an
    \(n\)-dimensional vector space. We say \(f\) is a
    \textbf{vector bundle} if \(\psi_x\) satisfies
    \(\psi_x \vert_{x_1} : f^{-1}(x_1) \rightarrow {x_1}\times V\) is a linear
    isomorphism for any \(x_1 \in U_x\).
\end{defn}

\begin{example}
    A mechanical system can be modelled by a configuration manifold
    \(\mathcal{Q} = \mathbb{R}^n \times (\mathbb{S}^1)^m\). At each
    \(q \in \mathcal{Q}\), the velocity of the system lies in the so-called
    tangent space \(T_q\mathcal{Q}\), which is an \(n+m\)-dimensional vector
    space. 
    The collection of tangent spaces
    \(T\mathcal{Q} := \left\{(q,v) \mid q \in \mathcal{Q}, v \in
    T_q\mathcal{Q}\right\}\) is called the \textit{tangent bundle} of
    \(\mathcal{Q}\). The natural
    projection \(\pi : T\mathcal{Q} \rightarrow \mathcal{Q}\) given by 
    \(\pi(q,v) = q\) is a vector bundle, because the fibers 
    \(\pi^{-1}(q) = T_q\mathcal{Q}\) are isomorphic to 
    \({q} \times \mathbb{R}^{n+m}\).
\end{example}

\begin{defn}[Cross-sections \cite{topology-fiber-bundles}]
    Let \(F \rightarrow E \xrightarrow[]{f} X\) be a fiber bundle.
    A \textbf{cross-section} is a continuous map \(\sigma : X \rightarrow E\)
    such that \(f \circ \sigma = \text{id}_X\).
\end{defn}

\begin{defn}[Bundle maps \cite{topology-fiber-bundles}]
    Let \(F_1 \rightarrow E_1 \xrightarrow[]{f_1} X_1\) and 
    \(F_2 \rightarrow E_2 \xrightarrow[]{f_2} X_2\) be fiber bundles. A
    continuous function \(\phi : E_1 \rightarrow E_2\) is a \textbf{bundle map}
    if there is a continuous function \(g : X_1 \rightarrow X_2\) so that
    \(g \circ f_1 = f_2 \circ \phi\).
\end{defn}

\begin{defn}[Pullback \cite{topology-fiber-bundles}]
    Let \(F \rightarrow E \xrightarrow[]{f} X\) be a fiber bundle. Let \(Y\) be
    a topological space and \(g : Y \rightarrow X\) a continuous function. The
    \textbf{pullback bundle over \(Y\)} is 
    \(g^\star(E) := \left\{ (y,e)\in Y \times E \mid g(y) = f(e)\right\}\). The
    natural projection \(\pi_Y : g^\star(E) \rightarrow Y\) with 
    \((y,e) \mapsto y\) is a fiber bundle with fiber \(F\).
\end{defn}

\begin{defn}[Manifolds \cite{intro-top-manifolds}]
    A topological space \(M\) is a manifold of dimension \(n\) if for any
    \(p \in M\) there exists an open neighbourhood \(U \subset M\) of \(p\)
    which is homeomorphic to an open subset of \(\mathbb{R}^n\).
\end{defn}

\begin{defn}[Submersions \cite{robots-fiber-bundles}]
    Let \(M\) and \(N\) be manifolds. A map \(h : M \rightarrow N\) is a
    \textbf{submersion} if it contains no singular points (that is, its Jacobian
    is always full rank).
\end{defn}

\begin{defn}[Groups \cite{intro-top-manifolds}]
    A \textbf{group} is a set \(G\) and an operation 
    \(\cdot : G\times G \rightarrow G\) with \((g,h) \mapsto gh\), along with
    the following axioms:
    \begin{enumerate}
        \item For all \(g,h,k \in G\), \((gh)k = g(hk)\).
        \item There exists an identity \(e \in G\) so that for all \(g \in G\)
            we have \(eg = ge = g\).
        \item For all \(g \in G\) there is an inverse \(h \in G\) so that
        \(gh = e\)
    \end{enumerate}
    If additionally the group satisfies \(gh = hg\) for all \(g,h \in G\), the
    group is called \textbf{abelian}.
\end{defn}

\begin{defn}[Group Homomorphism \cite{intro-top-manifolds}]
    Let \(G\) and \(H\) be groups. A function \(f : G \rightarrow H\) is a
    \textbf{homomorphism} if \(f(g_1g_2) = f(g_1)f(g_2)\) for all 
    \(g_1,g_2 \in G\).
\end{defn}

\begin{defn}[Group Torsion \cite{intro-top-manifolds}]
    An abelian group \(G\) is a \textbf{torsion group} if for each \(g \in G\),
    there exists \(n \in \mathbb{N}\) so that \(g^n = e\). If this is not the
    case, \(G\) is said to be \textbf{torsion-free}.
\end{defn}

%TODO: Fill this in and citation
\begin{defn}[Homotopy-Exact Sequence \cite{}]

\end{defn}

%TODO: Fill this in
\begin{defn}[Homology \cite{intro-top-manifolds}]

\end{defn}

%TODO: Cover Ehresmann's Theorem (see Robots and Fiber Bundles)

\section{Topology and the Robot Arm}
Summary of the paper in my own words, with plain English, and diagrams using the
2-link planar manipulator with configuration space T2.

The goal of the paper is to find a cross-section which is the map
that solves the inverse kinematics problem while avoiding singularities.
%---------- Bibliography ----------%
%\newpage
% This adds a line for the Bibliography in the Table of Contents.
\addcontentsline{toc}{chapter}{Bibliography}
\printbibliography
\end{document}
%/========== /Main Document ==========/%

% vim: set tw=80 ts=4 sw=4 sts=0 et ffs=unix :
