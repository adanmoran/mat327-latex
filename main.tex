%! TEX program = lualatex

% This project should be between 5-6 pages.
% No cover page required.
% Font should be 12pt, with 3/4" margins on all sides.
% It should be properly written, divided into sections, 
% and must contain a proper introduction and bibliography.

\documentclass[12pt]{article}
%/========== Preamble ==========/%
% Set Margins to 3/4"
\usepackage[margin=0.75in]{geometry}

%---------- Bibliography Style ----------%
\usepackage[style=ieee,backend=biber]{biblatex}
\addbibresource{bib.bib}

%/========== Main Document ==========/%
\begin{document}
% TEST CITATIONS
\cite{program-kin-red-manips}
\cite{robots-fiber-bundles}
\cite{topology-fiber-bundles}
\cite{topology-robot-arm}
    
\section{Introduction}
What is the inverse kinematics problem? Why do we care? What is usually done by
roboticists? How can Topology help us do this differently?

\section{Relevant Background}
Define the following, and when possible give examples.
\begin{itemize}
    \item fiber bundles and locally trivial fibrations
    \item bundle maps
    \item submersions
    \item homotopies
    \item homotopy-exact sequences
    \item homologies
    \item group torsion
    \item group generators
    \item cross-sections
    \item pullback bundle
\end{itemize}
Also cover \textbf{Ehresmann's Theorem}.

\section{Topology and the Robot Arm}
Summary of the paper in my own words, with plain English, and diagrams using the
2-link planar manipulator with configuration space T2.
%---------- Bibliography ----------%
%\newpage
% This adds a line for the Bibliography in the Table of Contents.
\addcontentsline{toc}{chapter}{Bibliography}
\printbibliography
\end{document}
%/========== /Main Document ==========/%

% vim: set tw=80 ts=4 sw=4 sts=0 et ffs=unix :
