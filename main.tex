%! TEX program = lualatex

% This project should be between 5-6 pages.
% No cover page required.
% Font should be 12pt, with 3/4" margins on all sides.
% It should be properly written, divided into sections, 
% and must contain a proper introduction and bibliography.

\documentclass[12pt]{article}
%/========== Preamble ==========/%
% Set Margins to 3/4"
\usepackage[margin=0.75in]{geometry}

%---------- Math Packages ----------%
\usepackage{amsmath}
\usepackage{amssymb}
\usepackage{mathtools}
\usepackage{amsthm}

% Define theorem styles
\newtheorem{thm}{Theorem}

\theoremstyle{definition}
\newtheorem{defn}{Definition}
\newtheorem{example}{Example}

%---------- Bibliography Style ----------%
\usepackage[style=ieee,backend=biber]{biblatex}
\addbibresource{bib.bib}

%/========== Main Document ==========/%
\begin{document}
% TEST CITATIONS
\cite{program-kin-red-manips}
\cite{robots-fiber-bundles}
\cite{topology-fiber-bundles}
\cite{topology-robot-arm}
    
\section{Introduction}
One common approach to controlling robotic manipulators 
(known colloquially as ``robot arms") is to pick a pose of the end effector and
reverse-engineer the joint angles required to reach this pose. This is known as
the \textbf{inverse kinematics} problem. For an arm with \(n\) revolute joints, 
one can model the desired end effector position and orientation by
\(x \in \mathbb{R}^3 \times SO(3)\) and the angles of the joints by 
\(\theta \in \mathbb{T}^n\). The inverse kinematics problem requires solving the
equation \(f(\theta) = x\) for \(\theta\), where \(f\) is an 
appropriately-defined function \cite{program-kin-red-manips}.

\begin{example}
%TODO: the 2D planar manipulator image.

The 2D planar manipulator with two revolute joints of lengths \(r_1\)
and \(r_2\) (Figure ????) whose end-effector is fixed in orientation with the
second joint has position \((x,y) \in \mathbb{R}^2\) and joint angles 
\(\theta = (\theta_1,\theta_2) \in \mathbb{T}^2\). 
The relation connecting these two is given by (\ref{eqn:2d-planar-manip}).
\begin{equation}\label{eqn:2d-planar-manip}
    \begin{bmatrix}x \\ y \end{bmatrix}
    = \begin{bmatrix} 
        r_1\cos(\theta_1) + r_2\cos(\theta_1 + \theta_2) \\
        r_1\sin(\theta_1) + r_2\sin(\theta_1 + \theta_2)
    \end{bmatrix} =: f(\theta)
\end{equation}
\end{example}
Unfortunately, there always exist joint angles where the relationship 
\(x = f(\theta)\) breaks down. These points are called 
\textbf{singular configurations}. If the dimension of the workspace is 
\(k > 0\), the singular configurations are characterized by the set
\[
    S = \left\{ \theta_0 \in \mathbb{T}^n \mid 
    \text{rank}\left\{
        \frac{\partial f}{\partial\theta}(\theta_0) 
    \right\} < k \right\}
\]

What is the inverse kinematics problem? Why do we care? What is usually done by
roboticists? How can Topology help us do this differently?

\section{Relevant Background}
Define the following, and when possible give examples.
\begin{itemize}
    \item fiber bundles and locally trivial fibrations
    \item bundle maps
    \item submersions
    \item homotopies
    \item homotopy-exact sequences
    \item homologies
    \item group torsion
    \item group generators
    \item cross-sections
    \item pullback bundle
\end{itemize}
Also cover \textbf{Ehresmann's Theorem}.

\section{Topology and the Robot Arm}
Summary of the paper in my own words, with plain English, and diagrams using the
2-link planar manipulator with configuration space T2.
%---------- Bibliography ----------%
%\newpage
% This adds a line for the Bibliography in the Table of Contents.
\addcontentsline{toc}{chapter}{Bibliography}
\printbibliography
\end{document}
%/========== /Main Document ==========/%

% vim: set tw=80 ts=4 sw=4 sts=0 et ffs=unix :
